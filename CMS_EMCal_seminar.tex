\documentclass[10pt]{beamer}

\usetheme[progressbar=frametitle]{metropolis}
\usepackage{appendixnumberbeamer}

\usepackage{booktabs}
\usepackage[scale=2]{ccicons}

\usepackage{pgfplots}
\usepgfplotslibrary{dateplot}

\usepackage{xspace}
\newcommand{\themename}{\textbf{\textsc{metropolis}}\xspace}

\usepackage{subfig}

\title{CMS Electromagnetic Calorimeter}
\subtitle{Design and Upgrade for HLLHC}
\date{\today}
\author{Daniele Brambilla}
\institute{University of Milano Bicocca}


\begin{document}

\maketitle

\begin{frame}{Table of contents}
  \setbeamertemplate{section in toc}[sections numbered]
  \tableofcontents%[hideallsubsections]
\end{frame}

\section[Introduction]{The Large Hadron Collider and the CMS Experiment}

\begin{frame}[fragile]{The LHC}
    \begin{figure}
        \centering
        \includegraphics[width=.85\textwidth]{./img/CERN_LHC.jpg}
        \caption{inserire didascalia}
        \label{fig:cern_lhc}
    \end{figure}
\end{frame}

\begin{frame}[fragile]{CMS}
  \begin{figure}
        \centering
        \includegraphics[width=\textwidth]{./img/CMS_front.jpg}
        \caption{CMS experiment section view}
        \label{fig:cms_front}
    \end{figure}
\end{frame}

\begin{frame}[fragile]{CMS}
  \begin{figure}
        \centering
        \includegraphics[width=.8\textwidth]{./img/CMS_scheme.png}
        \caption{CMS experiment scheme}
        \label{fig:cms_scheme}
    \end{figure}
\end{frame}


\section{EM Calorimeter}

\begin{frame}[fragile]{EM Calorimeter}



    \textbf{Barrel ECAL}
    \begin{itemize}
        \item  covering $|\eta| \leq 1.479 $ range
        \item $61200$ crystals organized in $5\times5$ modules and $36$ supermodules
        \item $360$-fold in $\phi$ & $(2\times85)$-fold in $\eta$
        \item crystal lenght: $230$\,mm corresponding to $25.8\,X_0$ 
    \end{itemize}
    \textbf{Endcap ECAL} 
    \begin{itemize}
        \item $1.479 \leq |\eta| \leq 3.0 $
        \item Each endcap divided in 2 "Dees", each with 3662 crystals organized in $5\times5$ supercrystals
        \item crystal lenght: $220$\,mm corresponding to $24.7\,X_0$ 
    \end{itemize}
\end{frame}

\begin{frame}{EM Calorimeter Design}
    \begin{figure}
        \centering
        \includegraphics[width=\textwidth]{./img/EMCal_Scheme.png}
        \caption{Emcal Scheme}
        \label{fig:emcalScheme}
    \end{figure}
\end{frame}

\begin{frame}[fragile]{Barrel EM Calorimeter}
  non mi piace molto questa foto, vedi se ne trovi una migliore...
  \begin{figure}
        \centering
        \includegraphics[width=.85\textwidth]{./img/ecal_barrel_photo.jpg}
        \caption{inserire didascalia}
        \label{fig:cms_barrel_photo}
    \end{figure}
\end{frame}

\begin{frame}{Endcap EM Calorimeter}
    \begin{figure}
        \centering
        \includegraphics[width=.85\textwidth]{./img/ecal_endcap_photo.jpg}
        \caption{One of the four "Dees" of the endcap EmCal }
        \label{fig:ecal_dees}
    \end{figure}
\end{frame}

%\begin{frame}{Design}
%    non mi piace molto questa immagine.. ma è un supermodulo del barrel
%    \begin{figure}
%        \centering
%        \includegraphics[width=100pt]{./img/emcal_module.png}
%        \caption{EmCal module}
%        \label{fig:emcalModule}
%    \end{figure}
%\end{frame}

\begin{frame}{Lead Tungstate ($\text{PbWO}_4$) Crystals}
    \begin{columns}
        \begin{column}[l]{0.40\textwidth}
            \begin{figure}
                \includegraphics[width=\textwidth]{./img/crystals.jpg}
            \end{figure}    
%            \begin{figure}
%                \includegraphics[width=\textwidth]{./img/crystal_production.jpg}
%            \end{figure}    
            \end{column}
        \begin{column}[l]{0.55\textwidth}
        
        \begin{itemize}
                \item $\rho = 8.3\,$g/cm$^3$
                \item $X_0 = 0.89\,$cm
                \item Moliere Radius $ = 2.2$\,cm
                \item Light Output: $4.5$ ph/MeV
                \item Green-Blue light, max @ 420\,nm
                \item Polished for internal reflection
            \end{itemize}
        \end{column}
    \end{columns}
    \bigskip
    \textbf{High radiation levels} throughout the duration of the experiment $\rightarrow$ wavelength dependent loss of light transmission without changes to the scintillation mechanism.
    
    Radiation hardness properties are required: the induced light attenuation length must be always greater than 3$\times$ crystal length. 
    
    Damage is tracked and corrected by a laser light monitoring system.
 
\end{frame}

\begin{frame}{Photodetectors}
    \textbf{Barrel EMCal}
    \begin{itemize}
        \item Reverse structure avalanche photodiodes (APDs)
        \item Glued to the back of the crystals
        \item High quantum efficiency ($\sim 75$\,\%) with mean gain of $50$
    \end{itemize}{}
    
    \textbf{Endcap EMCal}
    \begin{itemize}
        \item Vacuum Phototriodes
        \item Essentially photomultipliers, with a single gain stage
        \item Specially designed to withstand the $4$\,T magnetic field
        \item $22$\,\% quantum efficiency with mean gain of $10.2$ at $0$\,T
    \end{itemize}
\end{frame}

\begin{frame}{Pre-shower Detector}
    \emph{Sampling} calorimeter with two layers: lead radiators with silicon strip sensors placed in between.
    Located in the \emph{forward region}, where the angle between couples of photons is more likely to be small, due to the boost of the $\pi_0$.
    
    Main purposes:
    \begin{itemize}
        \item principal aim: the identification of neutral pions (from their decay products $\pi_0 \rightarrow \gamma \gamma$) within a fiducial region $1.653 < |\eta| < 2.6 $  
        \item improve position determination of e and $\gamma$ due to its finer granularity (silicon strips $p\simeq2\,$mm).
        \item help electron identification against minimum ionizing particles.
    \end{itemize}{}
    
\end{frame}

\begin{frame}{Elettronica e Segnale}
    voglio dire qualcosa di elettronica ed elaborazione del segnale? è necessario?
\end{frame}

\begin{frame}{Energy Resolution}
    Showers in EMCal are reconstructed by building \emph{clusters} of crystals. Best performance is obtained using a simple 3x3 (or 5x5) sliding window centered in the crystal having the maximum energy deposition.
    \begin{figure}
        \centering
        \includegraphics[height=130pt]{./img/resolution_1.png}
        \caption{Energy Distribution reconstructed during the test beam (pointed to the centre of the supermodule).}
        \label{fig:res1}
    \end{figure}{}
\end{frame}

\begin{frame}{Energy Resolution}
    \begin{figure}
        \centering
        \includegraphics[height=150pt]{./img/resolution_2.png}
        \caption{Energy distribution reconstructed during the test beam (pointed to a \emph{corner} of the supermodule). A single correction function, parametrized from the data, was applied to all regions of the supermodule to take into account variations in shower containment.}
        \label{fig:res2}
    \end{figure}{}
\end{frame}

\begin{frame}{Energy Resolution}
    Energy Resolution can be parametrized as a function of energy
    \begin{equation}
        \biggl(\frac{\sigma}{E}\biggr)^2 = \biggl(\frac{S}{\sqrt{E}}\biggr)^2 + \biggl(\frac{N}{E}\biggr)^2 + C^2 ,
    \end{equation}
  
    \begin{columns}
        \begin{column}[l]{0.45\textwidth}
        \begin{itemize}
            \item S is the \textbf{stochastic} term
            \item N is the \textbf{noise}
            \item C is a \textbf{constant} term
        \end{itemize}
        \end{column}
        \begin{column}[l]{0.45\textwidth}
            \begin{figure}
            \includegraphics[width=\textwidth]{./img/res_energy.png}
            \end{figure}
        \end{column}
    \end{columns}
    
\end{frame}

\begin{frame}{Calibration}
    It is a \emph{severe technical challenge}.
    Naturally divided in two parts:
    \begin{itemize}
        \item \textbf{Absolute energy scale}
        \item \textbf{Inter-calibration} Needed since single crystals have different scintillation light yield ($\sim 8\%$ RMS)
    \end{itemize}
    The final energy measurement is given by
    \begin{equation}
        E_{e,\gamma} = G \times \mathcal{F} \times \sum_i c_i \times A_i
    \label{eq:calib}
    \end{equation}
    where
    \begin{itemize}
        \item $G$ is a \emph{global absolute scale}
        \item $\mathcal{F}$ is a \emph{correction function} depending on particle type, position, $\eta$, momentum...
        \item $c_i$ are the \emph{inter-calibration coefficients}
        \item $A_i$ are the \emph{signal amplitudes} summed over the cluster of crystals 
    \end{itemize}{}
    
\end{frame}

\begin{frame}{Calibration}
    Various methods are used to obtain values for the parameters in equation \eqref{eq:calib}. These methods include:
    
    \begin{itemize}
        \item \textbf{Testbeam} precalibration.
        \item \textbf{Lab. Measurements} of the crystals light yield, giving a first estimate of the $c_i$ coefficients.
        \item \textbf{Phi independence} Taking advantage of the $\phi$ symmetry of deposited energy to inter-calibrate crystal rings at constant $\eta$.
        \item \textbf{Single electrons} Exploiting single electrons $p$ measurements from the \emph{tracker} to inter-calibrate different crystals in a single module.
        \item \textbf{Z $\rightarrow$ ee} Reconstruction of the ee invariant mass and calibration exploiting the Z mass constraint, studying the distribution of 
        \begin{equation}
            \epsilon^i = \frac{1}{2} \biggl[\biggl(\frac{M^i_\text{inv}}{M_Z}\biggr)^2 - 1\biggr]
        \end{equation}
    \end{itemize}
    
\end{frame}

\begin{frame}{Phi Independence Method}
    \begin{figure}
        \centering
        \includegraphics[width=\textwidth]{./img/intercalib_phi.png}
        \caption{Intercalibration precision as function on $\eta$, for barrel and endcap, for the \emph{phi-indipendence} method.}
        \label{fig:intercalib_phi}
    \end{figure}
\end{frame}



\begin{frame}{Single Electrons Method}
    \begin{figure}
        \centering
        \subfloat[][]{
            \includegraphics[width=.52\textwidth]{./img/single_e.png}
            } 
        \quad
        \subfloat[][]{
            \includegraphics[width=.37\textwidth]{./img/single_e_endcap.png}
        }
        \caption{Calibration precision vs $\eta$ obtained with the single electrons method. (a) barrel case (b) endcap case.}
        \label{fig:single_electron}
    \end{figure}

\end{frame}

\begin{frame}{Z $\rightarrow$ ee Method}
    \begin{figure}
        \centering
        \includegraphics[width=.75\textwidth]{./img/Zee_corr_factor.png}
        \caption{Correction factor $\mathcal{F}$ \eqref{eq:calib} in dependence of $ \eta$, as extracted with the Zee method. \emph{Golden} comes from a MC simulation, \emph{showering} is obtained from $2\,\text{fb}^{-1}$ of Z $\rightarrow$ ee with intercalibration precision at $2\%$.}
        \label{fig:Zee_correction_factor}
    \end{figure}
\end{frame}

\section{HL-LHC Upgrade}

\begin{frame}{High Luminosity LHC Upgrade}
    Objective: reach Integrated luminosity 3000\,fb^{-1}
    
    significative challenges due to High radiation environment and higher pileup
    
    various upgrades to the detector will be necessary
    
\end{frame}

\begin{frame}{Calorimeter upgrade}
    In particular also the calorimeters must be upgraded
    
    endcap calorimeter is where the radiations are higher, so it will be entirely replaced 
    
    \begin{figure}
        \centering
        \includegraphics{}
        \caption{Caption}
        \label{fig:my_label}
    \end{figure}{}
    
\end{frame}

\begin{frame}{High Granularity Calorimeter (HGCal)}
     It features unprecedented transverse and longitudinal segmentation for both electromagnetic (ECAL) and hadronic (HCAL) compartments.

    fine structure of showers can be measured and used to enhance pileup rejection and particle identification, whilst still achieving good energy resolution
    
    The ECAL and a large fraction of HCAL will be based on hexagonal silicon sensors of 0.5 - 1 cm2 cell size, with the remainder of the HCAL based on highly-segmented scintillators with SiPM readout.
    
     high-precision timing capabilities of the silicon sensors will add an extra dimension to event reconstruction, especially in terms of pileup rejection
     
     Algorithms are being developed to make 3d spatial reconstruction
     5D imaging calorimeter (x,y,z,t,E)
    
\end{frame}
\begin{frame}{Conclusion?????}
    \begin{itemize}
        \item one
        \item two
    \end{itemize}
\end{frame}
 
\end{document}
        \caption{Emcal Scheme}

